\documentclass[10pt]{article} %% What type of document you're writing.

 
\usepackage{graphicx}

 
\usepackage{hyperref}

 
\usepackage[dvipsnames]{xcolor}

\usepackage[utf8]{inputenc}

 
%%%%% Preamble

%% Packages to use

 
\usepackage{amsmath,amsfonts,amssymb} %% AMS mathematics macros

%% Title Information.

 
\documentclass[10pt]{article} %% What type of document you're writing.

 
\usepackage{graphicx}

 
\usepackage{hyperref}

 
\usepackage[dvipsnames]{xcolor} 

 
\usepackage{amsmath,amsfonts,amssymb} %% AMS mathematics macros

 

%% Title Information.

 
\title{VISION Y ALCANCE - PAGINA WEB PARA CALCULAR LA ENERGIA. \\ \\ \\ \\}

 
\author{Giselle Mendoza Barradas  \\ \\ \\ \\ \\ Universidad Veracruzana.  \\ \\ \\ \\ \\ Facultad de Negocios y Tecnologías, campus Ixtac.  \\ \\ \\ \\ \\ Pruebas de Software. \\ \\ \\ \\ \\Catedrático: Doc. Adolfo Centeno Tellez \\ \\ \\ \\ \\
501 Ingeniría de Software. \\ \\ \\ \\ \\ }

 
%% \date{29 sep 2020} %% By default, LaTeX uses the current date

 
%%%%% The Document

 
\begin{document}
 
\maketitle


\section{Visión del producto.} 
\newline
\newline

\subsection*{Statement de Visión}

 
\textcolor{black}{Para esta página web, brindar un resultado correcto, confiable y seguro, evitando a nuestros usuarios realizar el cálculo de forma tradicional (a mano) de esta forma optimizar el proceso de resolver una formula.} 


 
\subsection*{Características Importantes}

 
 \textcolor{black}{Esta es una página web que contiene características que son importantes mencionar antes de usarlas, para tener un mejor panorama del producto de software final.}
 \begin{itemize}
    \item Etiqueta de presentacion del desarrollador.
    \item Etiqueta con titulo.
    \item Etiqueta con la formula para calcular la energía.
    \item Etiqueta con el valor de velocidad de la luz.
    \item Campo de texto para ingresar la masa.
    \item Botón para ejecutar el cálculo.
    \item Campo de texto para mostrar el resultado.
\end{itemize}

\subsection*{Suposiciones y dependencias}
\textcolor{black}{Supuestos:}
\begin{itemize}
    \item Se proporcionará calidad en su implementación.
    \item Suponemos que el desarrollo no se pasará del tiempo establecido.
    \item Se espera que sirva como herramienta de apoyo. 
\end{itemize}
\textcolor{black}{Dependencias:}
\begin{itemize}
    \item La aplicación web estará diseñada para su visualización en dispositivos que cuente con internet.
    \item Diseñada para ser una pagina web dinámica.
\end{itemize}
 
 \section{Alcance y limitaciones.}
 
 \subsection*{Especificacion del alcance del software}

\textcolor{black}{}
\begin{itemize}
    \item Implementación del módulo de interfaz gráfica, el cual será el medio de comunicación entre el usuario y el sistema.
    \item Implementacion del backend para su correcto funcionamiento.
    \item Implementacion del módulo de pruebas.
\end{itemize}

\section{Limitaciones y exepciones.}

\textcolor{black}{Algunas de las implementaciones que estarán fuera de este proyecto son las siguientes:
}
\begin{itemize}
    \item Prevenibles:	
    \begin{itemize}
         \item Mala administracion del tiempo.
         \item Desconocer de las herramientas a utilizar.
    \end{itemize}
    \item Durante el desarrollo del proyecto:
    \begin{itemize}
         \item Falsedad en mis habilidades para la elaboración del proyecto.
    \end{itemize}
    \item Carecimiento de apoyo:
    \begin{itemize}
         \item Falta de conocimiento de los interesados para definir claramente requerimientos.
         \item Compromiso para finalizar el proyecto
    \end{itemize}
\end{itemize}

 

 
\end{document}